\documentclass{article}

\title{Discrete Math Study Guide}
\author{Ziyad Rahman}
\date{}

\usepackage{hyperref}
\usepackage{amssymb}
\usepackage{babel}
\usepackage{amsthm}
\usepackage{siunitx}

\newtheorem{theorem}{Theorem}[section]
\newtheorem{corollary}{Corollary}[theorem]
\newtheorem{lemma}[theorem]{Lemma}

\begin{document}
\maketitle

\tableofcontents
\newpage

\section{Logical Symbols and Deductive Reasoning}
\subsection{Variables and Statements}
A \textbf{variable} is a symbol that stands in for some specific value, be it a
person, number, etc. \\

\noindent A \textbf{statement} is a something that may evaluate to true or false.
It is usually either in the form $P$ if it does not depend on a variable
or $P(x)$ if the statement's truth depends on what the input is.

\subsection{Connective Symbols}
\subsection{Logical Laws}
\subsubsection{Associative Law}
\begin{eqnarray}
    (P \land Q) \land R &=& P \land (Q \land R) \\
    (P \lor Q) \lor R &=& P \lor (Q \lor R)
\end{eqnarray}

\subsubsection{Communative Law}
\begin{eqnarray}
    P \land Q &=& Q \land P \\
    P \lor Q &=& Q \lor P
\end{eqnarray}
\subsubsection{Distributive Law}
\begin{eqnarray}
    P \land (Q \lor R) &=& (P \land Q) \lor (P \land R) \\
    P \lor (Q \land R) &=& (P \lor Q) \land (P \lor R)
\end{eqnarray}
\subsubsection{Double Negation Law}
\begin{eqnarray}
    \neg \neg P &=& P
\end{eqnarray}
\subsubsection{De Morgan's Law}
\begin{eqnarray}
    \neg(P \land Q) &=& (\neg P \lor \neg Q) \\
    \neg(P \lor Q) &=& (\neg P \land \neg Q)
\end{eqnarray}
\subsubsection{Idempotent Law}
\begin{eqnarray}
    P \land P &=& P \\
    P \lor P &=& P
\end{eqnarray}
\subsubsection{Absorption Law}
\begin{eqnarray}
    P \land (P \lor Q) &=& P \\
    P \land (P \lor Q) &=& P 
\end{eqnarray}

\subsection{Truth Tables}
Truth tables are are relative straight forward concept. The aim is
to evaluate the truth of statement by breaking it down into its
smallest parts, then seeing if the final statement is true or false
based on the truth of the sub-statements. Here is a simple example,
\begin{center}
    \begin{tabular}{c | c | c}
        P & Q & $P \lor Q$ \\
        \hline
        T & T & T \\
        T & F & T \\
        F & T & T \\
        F & F & F \\
    \end{tabular}
\end{center}

\subsection{The Conditional}
\subsubsection{Definition}
The conditional can be thought of as an "if, then" statement. It 
primarily demonstrates some relationship between two statements.
In symbols, it is represented as,

\[ P \rightarrow Q \]

\noindent This statement can be read several ways in English:
\begin{enumerate}
    \item $P$ implies $Q$
    \item $P$ only if $Q$
    \item $P$ is a sufficient condition for $Q$
    \item $Q$, if $P$
    \item $Q$ is a necessary condition for $P$
\end{enumerate}

\subsubsection{The Truth of a Conditional}
We can demonstrate the truth of the conditional via a truth table.
\begin{center}
    \begin{tabular}{c | c | c}
        P & Q & $P \rightarrow Q$ \\
        \hline
        T & T & T \\
        T & F & F \\
        F & T & T \\
        F & F & T \\
    \end{tabular}
\end{center}

To put the truth table into plain words, the conditional is
true only if both $Q$ is true or if $P$ and $Q$ are both false. In other
words, the conditional is only false if only $Q$ is false.

\subsubsection{The Conditonal in Logical Connectives}
We can write the conditional in terms of basic logical connectives.
The definition of conditional in these terms is as follows.

\[ P \rightarrow Q \; \; \; \equiv \; \; \; \neg P \lor Q \; \; \; \equiv \; \; \; P \land \neg Q\]

\noindent Note that the rightmost statement is the same as the middle statement, but De Morgan's Law
was applied.

We can verify that these statements are equivalent via another truth table.

\begin{center}
    \begin{tabular}{c | c | c | c | c}
        P & Q & $P \rightarrow Q$ & $\neg P \lor Q$ & $P \land \neg Q$ \\
        \hline
        T & T & T & T & T \\
        T & F & F & F & F \\
        F & T & T & T & T \\
        F & F & T & T & T \\
    \end{tabular}
\end{center}

\subsubsection{The Converse}
The converse of a conditional is simply the conditional, but the statements
have been swapped around.
\begin{eqnarray} \nonumber
    P \rightarrow Q &\not\equiv & Q \rightarrow P
\end{eqnarray}

We could write a truth table to demonstrate that these statements are \textbf{NOT}
equivalent, but we will use the definition of the conditional (the logical symbols
version) to demonstrate intuitively that these are not the same.

\begin{eqnarray} \nonumber
    P \rightarrow Q &\equiv& \neg P \lor Q \\
    Q \rightarrow P &\equiv& \neg Q \lor P \\
    \neg P \lor Q &\not\equiv& \neg Q \lor P
\end{eqnarray}

\subsubsection{The Contrapositive}
The contrapositive of a conditional is a negated version of the original
statement. Unlike the converse of conditional, the contrapositive is 
equivalent to the original statement.
\begin{eqnarray} \nonumber
    P \rightarrow Q &\equiv& \neg Q \rightarrow P
\end{eqnarray}

We could use a truth table to show that these statements are equivalent,
but we can also use the logical forms of the conditionals achieve the same end.

\begin{eqnarray} \nonumber
    P \rightarrow Q &\equiv& \neg P \lor Q \\ \nonumber
    \neg Q \rightarrow P &\equiv& \neg \neg Q \lor \neg P \\ \nonumber
    \neg P \lor Q &\equiv& \neg \neg Q \lor P
\end{eqnarray}

\subsection{The Biconditional}
\subsubsection{Definition}
The biconditional is often read as "if an only if". It can be written in terms
of the conditional or logical connectors. It is written as follows.
\begin{eqnarray} \nonumber
    P \leftrightarrow Q
\end{eqnarray}

In terms of the conditional,
\begin{eqnarray}
    P \leftrightarrow Q  &\equiv& (P \rightarrow Q) \land (Q \rightarrow P)
\end{eqnarray}

The final definition is in terms of logical connectors.
\begin{eqnarray}
    P \leftrightarrow &\equiv& (\neg P \lor Q) \land (\neg Q \lor P)
\end{eqnarray}

\noindent Any statements that are equivalent to those above are valid definition of
the biconditional.

\subsubsection{The Truth of a Biconditional}
We can determine the truth of a biconditional via a truth table.

\begin{center}
    \begin{tabular}{c | c | c}
        P & Q & $P \leftrightarrow Q$ \\
        \hline
        T & T & T \\
        T & F & F \\
        F & T & F \\
        F & F & T \\
    \end{tabular}
\end{center}

As we can see, the biconditional only evaluates to true if (and only if) both statements
involved are true.

\subsection{Arguments}

\section{Quantifiers}
\subsection{Motivating Quantifiers}
\subsection{The Universe of Discourse}
\subsection{The Universal Quantifier}
\subsection{The Existential Quantifier}
\subsubsection{Uniqueness}
\subsection{Bound Variables}
\subsection{Quantifier Negation}

\section{Set Theory}
\subsection{Defining Sets}
\subsubsection{Important Sets}
\subsubsection{Truth Sets}
\subsection{Basic Set Operations}
\subsection{Index Sets}
\subsection{Families of Sets}
\subsubsection{The Power Set}
\subsection{Operations on Families of Sets}

\section{Introductory Proof Strategies}
\subsection{Theorems, Propositions, and Lemmas}
\subsection{Proof Writing Basics}
\subsection{Direct Proofs}
\subsection{Proof by Contrapositive}

\end{document}