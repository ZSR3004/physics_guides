\section{Prerequisite Math}

Obviously physics needs a lot of math. This section covers is mostly about vector math you'll need for this course.

\subsection{Special Angles}

Here are very common angles that may be asked of you. It's best to memorize these angles and their
values when trigonometric functions are used on them. These angles are reference angles, and because
of that, they only represent magnitude. You need to add the sign after you are done computing the value based
on what is physically happening in the scenario/question.

\begin{center}
\begin{tabular}{| c | c | c | c | c | c |}
    \hline
    Operation & $\ang{0} = 0 $ & $\ang{30} = \frac{\pi}{6}$ & $\ang{45} = \frac{\pi}{4}$ & $\ang{60} = \frac{\pi}{3}$ & $\ang{90} = \pi$ \\
    \hline
    $\sin (\theta)$ & 0 & $\frac{1}2$ & $\frac{\sqrt{2}}{2}$ & $\frac{\sqrt{3}}{2}$ & $1$ \\
    \hline
    $\cos (\theta)$ & 1 & $\frac{\sqrt{3}}{2}$ & $\frac{\sqrt{2}}{2}$ & $\frac{1}2$ & $0$ \\
    \hline
    $\tan (\theta)$ & 0 & $\frac{1}{\sqrt{3}}$ & $1$ & $\sqrt{3}$ & undefined \\
    \hline
\end{tabular}
\end{center}

\subsection{Basic Vector Operation}
A vector is a way to store numbers. In a physics sense, it's really just an arrow pointing from one place to another. Below are vector basics
including how to add and multiply vectors. This covers all important vector operations done in this course. A normal number like \( 5 \) is
called a scalar. In this class if something is not a vector, then it is a scalr.

\subsubsection{Representing Vectors and Magnitude}
Let $\vec{A}$ be a vector of 3 dimensions.
\begin{equation}
    \vec{A} =  A_x \hat{i} + A_y \hat{j} + A_z \hat{k}
\end{equation}
There are other ways to represent vectors, but this is the way we'll do it in this class. Each \(A_{something}\) literally just represents the
$x$, $y$, or $z$ coordinate of the arrow's tip.

A quick notational thing is the difference between \( \vec{|A|} \) and \( \vec{A} \). The first one represents the magnitude while the second one
is the actual vector. If you think about our arrow representation, the magnitude of a vector is literally just how long it is. In a 2D space,
if you think about a triangle, think about it as finding the hypotenuse from the length of the base and height. In fact, to find the magnitude
of a vector from its normal form, you just use pythagorean's theorem. Below is that theorem in three dimensions.
\begin{equation}
    |\vec{A}| = \sqrt{A_x^2 + A_y^2 + A_z^2} 
\end{equation}

\subsubsection{Vector Addition and Subtraction}
\begin{equation}
    \vec{A} \pm \vec{B} = \langle \vec{A}_x \pm \vec{B}_x, \; \vec{A}_y \pm \vec{B}_y, \; \vec{A}_z \pm \vec{B}_z \rangle = \vec{C}
\end{equation}

\subsubsection{Vector Multiplication}
Vector multiplication comes in three flavors: multiplication by a scalar, the dot product, and the cross product. Scalar multiplication is multiplication
between a scalar and a vector. The other two occur between two vectors. On a super high level, the dot product results in a scalar, whereas the 
cross product creates another vector.

\noindent \textbf{Scalar Multiplication.} Multiplying a vector by a scalar is by far the easiest vector multiplication. Suppose $n$ is a scalar (that
is, $n$ is some number).
\begin{equation}
    n\vec{A} = (A_x \times n) \hat{i} + (A_y \times n) \hat{j} + (A_z \times n) \hat{k}
\end{equation}
You basically just distribute it over the vector.

\noindent \textbf{The Dot Product.}
There are two ways to calculate the dot product.
\begin{equation}
    \vec{A} \cdot \vec{B} = (\vec{A_x} \times \vec{B_x}) + (\vec{A_y} \times \vec{B_y}) + (\vec{A_z} \times \vec{B_z})
\end{equation}
\begin{equation}
    \vec{A} \cdot \vec{B} = |\vec{A}| \times |\vec{B}|\cos{\theta}
\end{equation}

This is the second equation solved for theta, which often comes in handy.
\begin{equation}
    \theta = \cos^{-1} \left( {\frac{\vec{A} \cdot \vec{B}}{|\vec{A}| \times |\vec{B}|}} \right)
\end{equation}

\noindent \textbf{The Cross Product.}
There are also two ways to calculate the cross product. A quick note: when you do the cross product of two 3D vectors, your resultant vector (the
output) will be one dimensions smaller. 

\begin{equation}
    \vec{A} \times \vec{B} = |\vec{A}||\vec{B}| \sin \theta
\end{equation}

\begin{equation}
    \vec{A} \times \vec{B} = (A_y \times B_z - A_z \times B_y)\hat{i} + (A_z \times B_x - A_x \times B_z)\hat{j} + (A_x \times B_y - A_y \times B_x)\hat{k} \\
\end{equation}
This formula will be used a lot less. In fact, you'll almost never use it.