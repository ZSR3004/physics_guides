\section{Kinematics}
Kinematics refers to the process of figuring out how something moves over time, including its velocity and acceleration.

\subsection{Primer: Translational and Rotational Motion}
Before discussing how to actually do kinematics, we'll start by talking about the two types of motion studied in this class: translational and rotational.
\textbf{Translational} motion refers to things moving in a line or curve. For example, a car driving down the highway or a ball being thrown between two 
people. \textbf{Rotational} motion refers to the motion of something spinning such as someone on a merry go round or a pulley. 

Something can have one and not the other, both, or neither. If I have a bowling ball in my hand and I'm just holding it, then it is stationary, so
it has no motion. If I begin to spin the bowling ball on the floor and it stays in place, then it has rotational but not translational motion. If I
slide it (that is, it's not rolling, the holes are always at the top for instance) then it has translational but not rotational motion. If I throw it
down the alley and its rolling as it barrels towards the bowling pins, then it has both types of motion.

\subsection{Primary Kinematic Equations}
We start with $\vec{r}$ which represents position in a unit of distance (\textit{m}). Before we begin, this variable is a little weird. In most of the
equations given, you won't see a $\vec{r}$ because things will be generalized in one dimension. Just understand that $\vec{r} = x \; \hat{i} + 
y \; \hat{j} + z \; \hat{k}$.

We can think about our kinematic equations in very general terms using derivatives. The core kinematic equations are as follow.
\begin{equation}
    \vec{v} = \frac{\mathrm{d}\vec{r}}{\mathrm{d}t}
\end{equation}
\begin{equation}
    \vec{a} = \frac{\mathrm{d}\vec{v}}{\mathrm{d}t}
    \end{equation}

\subsubsection{Special Case: Constant Acceleration} \label{kinematics const a}
Then, we have the constant acceleration equations which is what you'll use more often.
    \begin{equation}\label{eq: kinematic 1}
    \vec{v}_f = \vec{v}_i + \vec{a}t
\end{equation}
\begin{equation}\label{eq: kinematic 2}
    \vec{r}_f = \vec{r}_i + \vec{v}_it+ \frac{1}{2}\vec{a}t^2 
\end{equation}
\begin{equation}\label{eq: kinematic 3}
    \vec{v}_f^{ \: 2} = \vec{v}_i^{ \: 2} + 2 \vec{a} (\vec{r}_f - \vec{r}_i)
\end{equation}
\subsubsection{Relative Kinematics Based on Postion}
Since positions are relative, we can deduce the relative speed of an object \textit{A} relative to a Point \textit{B} if we know the speed of \textit{A} 
relative to a position \textit{C} and the speed of \textit{C} relative to \textit{B}'s position. This applies to position, velocity, and acceleration. 
To solve for this, we can use the following formula.
\begin{equation}
    v_{\frac{A}{B}} = v_{\frac{A}{C}} +  v_{\frac{C}{B}}
\end{equation}

\subsection{Rotational Kinematics}
Just a quick note before this section begins. Appendix \ref{Appendix B} contains a useful chart that summarizes all of the translation and rotational 
counterparts. The section below provides a slightly more in-depth look at the topic, including useful formulas, but if you're just trying to remember 
what the rotational counterpart to displacement is, you might be better off checking the chart.

\subsubsection{Introducing Rotational Kinematics} \label{trans to ang}
Rotational kinematics operate in practically the same way as translational kinematics, but deals with objects that are spinning or otherwise 
moving in a circular motion.
Below, we detail how you can take tangential measurements and turn them into their rotational counterparts. \\
\[ \vec{s} = r \theta \]
\[ \vec{v} = r \omega \]
\[ \vec{a} = r \alpha \]
\[ t = t \]

In these equations, $\vec{r} = \vec{s}$ for clarity's sake because the $r$ that is included actually refers to the path's radius. Both $\vec{r}$ and 
$\vec{s}$ mean position. You would refer to each counterpart as the "angular" version of the translational one 
(ex. displacement becomes angular displacement).

\subsubsection{Primary Rotational Kinematics Equations}
Since we have already defined $\theta$ as angular displacement (in radians) we can define the other kinematic values.
\begin{equation}
    \omega = \frac{\mathrm{d}\theta}{\mathrm{d}t}
\end{equation}
\begin{equation}
    \alpha = \frac{\mathrm{d}\omega}{\mathrm{d}t}
\end{equation}

\subsubsection{Special Case: Constant Rotational Acceleration}
Now, we can combine the special case kinematic equations found in \ref{kinematics const a} with the definitions that relate the translational and angular values in \ref{trans to ang}. The process is simple, since the angular form of a kinematic value is the just the translational counterpart divided by $r$, we can just divide each of the \ref{kinematics const a} equations by $r$ to create rotational counterparts.
\begin{equation}
    \omega_f = \omega_i + \alpha t
\end{equation}
\begin{equation}
    \theta_f = \theta_i + \omega_it+ \frac{1}{2}\alpha t^2 
\end{equation}
\begin{equation}
    \omega_f^{ \: 2} = \omega_i^{ \: 2} + 2 \alpha  (\theta_f - \theta_i)
\end{equation}

\subsubsection{Period and Frequency}
As an object rotates around itself (or travels in a circular path), it's bound to come back to the same point eventually. Note that this only works for 
zero acceleration otherwise the time to finish one rotation would keep changing and that defeats the purpose of calculating such a measurement in the 
first place.The time it takes for one full rotation (usually called a cycle) is called the \textit{period}, recorded in some unit of time. 
\begin{equation}
    T = \frac{2 \pi}{\omega}
\end{equation}
\noindent \textit{Frequency} is the inverse of the period, defined as the number of rotations an object makes in one unit of time. Below are 
definitions of both in relation to angular velocity and each other.
\begin{equation}
    f = \frac{1}{T}
\end{equation}
An important thing to note is the unit of frequency. Since it's the inverse of $T$, its units are technically just "cycle per unit time", but if the 
unit of time is seconds, then we the unit hertz (\textit{hz}). In other words $hz = \frac{1}{s}$.

\subsubsection{Centripedal Acceleration}
Centripetal acceleration, or rotational acceleration, is another important value that is unique to rotational motion. In effect, it is the inward 
acceleration that keeps an object that is moving in a circular path in that path. For example, you might be swinging a ball on a string in a circle; 
the reason it doesn't fly out is because the sting is applying a force (and therefore accelerating the ball) back into the circle as tangential 
velocity "tries to get it" to fly out. We can define centripetal acceleration using both tangential velocity and angular velocity.
\begin{equation}
    \vec{a}_c = \frac{\vec{v}^{\: 2}}{r} = \omega^2 r
\end{equation}
