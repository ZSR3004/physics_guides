\subsection{Simple Harmonic Motion (SHM)}
SHM is an extremely fundamental concept in physics. The two main types of SHM systems we will look at are mass-spring systems and simple pendulums. We'll
start by deriving the core equations for SHM since they'll come up again and again, then we'll talk about period and frequency. We didn't cover this
topic much when I took the class because we ran short on time, so this might not be as holistic as other sections.

\subsection{The Core Equations}
Imagine a mass-spring system on a surface with no friction. We can begin by writing Newton's
second law for this system (note that we are only concerned with the $x$ direction, so we only use scalars).
\begin{eqnarray*}
    F_{net} &=& ma \\
    F_{spring} &=& ma \\
    -kx &=& ma \\
    \frac{-kx}{m} &=& a \\
    \frac{-kx}{m} &=& \ddot{x}
\end{eqnarray*}
(We did a little notational trick here. $\dot{x}$ is the derivative of $x$ with respect to time, or velocity. $\ddot{x}$ is the derivative with respect
to time twice, or acceleration).

Now, we have a differential equation which we can solve. There are a number of answers, but the ones most important are defined by
\begin{eqnarray} \label{eq : SHM}
    x(t) &=& Acos(\omega t + \phi_0).
\end{eqnarray}

By taking the derivative, we can also talk about velocity and acceleration.

\begin{eqnarray} \label{eq : SHMav}
    \dot{x}(t) &=& -A \omega sin(\omega t + \phi_0) \\
    \ddot{x}(t) &=& -A \omega^2 cos(\omega t + \phi_0)
\end{eqnarray}

\subsection{Period and Frequency}
We can write the angular frequency as,
\begin{equation}
    \omega = w \pi f.
\end{equation}

Why, yes! It is confusing that it uses the $\omega$ symbol and has the units $\frac{rads}{s}$. No, it is not the same thing as angular frequency.
The two values are VERY closely related. I can't do the explanation justice because I don't fully understand it, so I won't try at risk of confusing you,
dear reader. Just know that this is a thing you'll need to calculate.

We can also determine the period of the two systems we're interested in. For a pendulum, we can record how long it takes to come back to the starting
point with the following equation.
\begin{equation}
    T = 2 \pi \sqrt{\frac{l}{g}}
\end{equation}

Where, $l$ is the length of the pendulum's string. We take $g$ as the standard gravitational acceleration constant, but note that it would be different if
the problem was on Mars or some other planet (since the gravity would impact the pendulum falling differently). Notice how mass does not impact the
period at all.

We can also calculate the period of a mass-spring system with the following equation.
\begin{equation}
    T = 2 \pi \sqrt{\frac{m}{k}}.
\end{equation}